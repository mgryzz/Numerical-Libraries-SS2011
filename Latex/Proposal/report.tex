\documentclass{article}

\usepackage{amsmath}
\usepackage[german]{babel}
\usepackage{amssymb}
\usepackage{amsxtra}
\usepackage[dvips]{epsfig,psfrag}
\usepackage{listings}

\newcommand{\refchapter}[1]{Kapitel~\ref{#1}}
\newcommand{\refsec}[1]{Sektion~\ref{#1}}
\newcommand{\refeqn}[1]{Gleichung~(\ref{#1})}
\newcommand{\reffig}[1]{Abbildung~\ref{#1}}

\title{
{\bf \scriptsize RHEINISCH-WESTF\"ALISCHE TECHNISCHE HOCHSCHULE AACHEN \\
LuFG Informatik 12 (Prof. Dr. rer. nat. Uwe Naumann)}
\vspace{.5cm} \\
\epsfig{file=figures/STCE_Logo_WWW.eps,width=.7\textwidth}
\vspace{1cm} \\
{\bf \Large Mustertitel} \\
{\large Seminararbeit} 
}

\author{vorgelegt von Uschi Musterfrau (Matr.-Nr. 0815)}

\begin{document}

\lstloadlanguages{[ISO]C++}
\lstset{basicstyle=\small, numbers=left, numberstyle=\footnotesize,
  stepnumber=1, numbersep=5pt, breaklines=true, escapeinside={/*@}{@*/}}


\pagestyle{headings}

\maketitle
\newpage

\section*{Erkl\"arung}

Hiermit versichere ich, dass ich die Arbeit selbst\"andig verfasst und keine
anderen als die angegebenen Quellen und Hilfsmittel benutzt sowie Zitate 
kenntlich gemacht habe. \\
\\
\\
\\
\\
\\
\\
\\
Uschi Musterfrau \\
\\
Ort, Datum.

\newpage
\section*{Danksagung}

{\em an Mamma, Papa, Freund(in), Betreuer(in), Hund, ... (falls zutreffend)}

\newpage
\tableofcontents
\newpage

\section{In die Breite gehend ...}
\label{ch:2}

{\em Titel anpassen}


\subsection{Aufgabenstellung und Struktur des Dokuments}
\label{sec:2.0}

\subsection{Grundlagen}
\label{sec:2.1}

\subsection{Stand von Forschung und Entwicklung}
\label{sec:2.2}

{\em Referenzen auf relevante Literatur (z.B. \cite{Ries1522Rad})}

\section{In die Tiefe gehend ...}
\label{ch:3}

{\em Titel anpassen}

\subsection{Theoetische Resultate}
\label{sec:3.1}

\subsection{Implementierung}
\label{sec:3.2}

\subsection{Anwendung}
\label{sec:3.3}

\section{Fazit}
\label{ch:4}

\subsection{Zusammenfassung}
\label{sec:4.1}

\subsection{Diskussion}
\label{sec:4.2}

\subsection{Ausblick}
\label{sec:4.3}

\nocite{}
\bibliographystyle{plain}
\bibliography{report}

\newpage
\appendix

\section{Benutzerdokumentation}
\label{app1}

{\em wohlstrukturierte und gut lesbare Dokumentation der entwickelten Software aus Nutzersicht (falls zutreffend)}

\section{Entwicklerdokumentation}
\label{app2}

{\em wohlstrukturierte und gut lesbare Dokumentation der Software aus 
Entwicklersicht; zahlreiche Referenzen in den
Quellcode in \refchapter{app4}} (falls zutreffend) \\
\\
Die Signatur der Funktion {\tt bar} finden Sie in Zeile~\ref{zn_bar} des Quellcodes~\ref{klasse1.1.hpp} in \refsec{ssec:6.1.1}.

\section{Fallstudien}
\label{app3}

{\em dokumentierte zus\"atzliche Beispielanwendungen der entwickelten Software}

\section{Quellcode}
\label{app4}

{\em einfach referenzierbare Version des Quelltexts} 

\subsection{Paket 1}
\label{sec:6.1}

\subsubsection{Klasse 1.1}
\label{ssec:6.1.1}

\begin{lstlisting}[caption=Dokumentierter Quellcode in {\tt klasse1.1.hpp},label=klasse1.1.hpp]
class foo {
  public:
    int bar(const float& f); /*@\label{zn_bar}@*/
}
\end{lstlisting}

\subsubsection{Klasse 1.2}

\subsection{Paket 2}

\subsection{Paket 3}

\end{document}

